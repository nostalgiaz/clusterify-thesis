\chapter{Clusters e clustering}
	Con il nome di \emph{clustering} si definisce la tecnica non supervisionata utile al raggruppamento di oggetti in insiemi secondo determinate regole. Gli elementi appartenenti allo stesso \emph{cluster} sono più simili tra loro rispetto a quelli contenuti negli altri gruppi. La \emph{cluster analysis} è diventata una componente molto importante nell'analisi dei dati, sia in campo scientifico che industriale. Quando un insieme di dati deve essere organizzato, questa tecnica assegna ogni singolo dato ad un gruppo utilizzando strutture presenti nei \emph{raw data}.

	L'apprendimento non supervisionato è una tecnica che consiste nel fornire al sistema una serie di dati in input che, successivamente, riclassificherà ed organizzerà sulla base di caratteristiche che questi oggetti hanno in comune senza avere bisogno di ulteriori informazioni.

	Queste regole tendono a minimizzare la \emph{distanza} interna a ciascun gruppo ed a massimizzare quella tra i gruppi stessi.

	La distanza è una qualsiasi funzione $d:X \times X \to \mathbb{R}$ che soddisfa:
	\begin{align*}
		d(x,y) &\geq 0 \\
		d(x,y) &= 0 \iff x=y \\
		d(x,y) &= d(y,x)
	\end{align*}

	La scelta della distanza influenza drasticamente la forma dei cluster poiché i rapporti tra le varie distanze possono cambiare totalmente. Tra le più comuni sono presenti la \emph{distanza euclidea} e la \emph{distanza di Manhattan}.

	Non esiste un algoritmo in grado di raggruppare assolutamente in modo migliore rispetto ad un altro, infatti esistono tecniche specifiche per ogni tipologia di dato, da quello statistico a quello sociale.

	Ci si può ispirare a molteplici nozioni per il raggruppamento: dalla creazione di gruppi caratterizzati da una piccola distanza tra i singoli membri, all’utilizzo di particolari distribuzioni statistiche.

	Molti degli algoritmi usati per fare clustering godono di alcuni tratti in comune che ci portano a definire dei \emph{modelli}.

	I tipici modelli di clustering sono:
	\begin{itemize}
		\item Partizionale
		\item Gerarchico
		\item Spectral
		\item Altri modelli
	\end{itemize}

	\section{Clustering partizionale}
	Gli algoritmi di clustering di questa famiglia creano una partizione delle osservazioni minimizzando la funzione di costo:
	\begin{equation*}
	  \sum_{i=1}^{k}{E(C_i)}
	\end{equation*}
	ove $k$ è il numero dei cluster richiesti in output, $C_i$ è l'$i$-esimo cluster e $E:C \rightarrow R^{+}$ è la funzione di costo associata al singolo cluster.

	Questa funzione viene spesso tradotta in\cite{funzione_costo}:
	\begin{equation*}
	  	E(C_i) = \sum_{j=1}^{|C_i|}{dist(x_j, center(i))}
	\end{equation*}
dove $|C_i|$ è il numero di oggetti presenti nel $i$-esimo cluster e $dist(x_j, center(i))$ è una funzione che calcola la distanza tra il punto $x_j$ ed il centro del cluster $i$-esimo.

	Questa tipologia di algoritmi solitamente richiede di specificare il numero di cluster distinti che si vogliono raggiungere a processo terminato e mira ad identificare i gruppi naturali presenti nel dataset, generando una partizione composta da cluster disgiunti la cui unione forma il dataset originale.

	Questo significa che ogni cluster ha almeno un elemento e che un elemento appartiene ad un solo cluster.

	Per segmentare l'insieme in sottogruppi si utilizza il concetto di centri: inizialmente sono posizionati in modo casuale, o secondo un qualsivoglia algoritmo, e iterativamente mossi fino a che questi non raggiungano uno stato di fermo. Quando ciò accade si è in grado di definire a quale centro appartiene il singolo punto e, di conseguenza, la struttura dei cluster calcolati. 

	Gli algoritmi più famosi appartenenti questa categoria sono: 
	\begin{itemize}
	  	\item K-means
	  	\item K-medoids
	  	\item Affinity Propagation
	\end{itemize}

	\section{Clustering gerarchico}
	Gli algoritmi di clustering gerarchico creano una rappresentazione gerarchia ad albero dei cluster.
	Le strategie sono tipicamente di due tipi: 
	\begin{itemize}
		\item Metodo agglomerativo \emph{(bottom-up)} \\
		Si inizia col creare ed associare un cluster ad ogni oggetto, e si procede poi con l'unione di questi, basando la selezione degli insiemi da unire su una \emph{funzione di similarità}.

		\item Metodo divisivo \emph{(top-down)} \\
		Partendo da un unico cluster contenente tutti gli oggetti, lo si divide basando la scelta della selezione dell’insieme da dividere su una \emph{funzione di similarità}.
	\end{itemize}

	Questa tipologia di algoritmi necessita anche di alcuni criteri di collegamento che specificano la dissimilarità di due insiemi utilizzando la distanza valutabile tra gli stessi. Questi criteri possono essere: 
	\begin{itemize}
		\item Complete linkage: distanza massima tra elementi appartenenti a due cluster
			\begin{equation*}
				\max \, \{\, d(a,b) : a \in C_1,\, b \in C_2 \,\}
			\end{equation*}
		\item Minimum o single-linkage: distanza minima tra elementi appartenenti a cluster diversi
			\begin{equation*}
			 	\min \, \{\, d(a,b) : a \in C_1,\, b \in C_2 \,\}
			\end{equation*}

		\item Average linkage: la media delle distanze dei singoli elementi
			\begin{equation*}
				\frac{1}{|C_1| |C_2|} \sum_{a \in C_1 }\sum_{ b \in C_2} d(a,b). 
			\end{equation*}
	\end{itemize}
	dove $C_1$ e $C_2$ rappresentano i due cluster da unire e $d$ la distanza prescelta.

	Gli algoritmi più famosi appartenenti questa categoria sono SLINK (single-linkage) e CLINK (complete-linkage)\cite{clustering_gerarchico}.

	\section{Clustering spectral}
	Questa tipologia di clustering nasce dal bisogno di colmare delle mancanze presenti negli altri metodi: i metodi partizionali riescono a creare cluster solamente di forma sferica, basandosi, come abbiamo già visto, sul concetto di centro; mentre i metodi density-based hanno un approccio troppo sensibile ai parametri dati.

	Per risolvere questi problemi si utilizza un grafo di similarità che ha come vertici le componenti da clusterizzare e, come valore degli archi, la similarità delle componenti tra cui questi sono sottesi.

	Una volta creato il grafo, si procede con una serie di tagli minimi che possono essere individuati con un "taglio normalizzato" secondo la seguente formula:

	\begin{equation*}
		NormCut(C_1, C_2) =  \frac{Cut(C_1, C_2)}{Vol(C_1)} + \frac{Cut(C_1, C_2)}{Vol(C_2)}
	\end{equation*}
	dove $C_1$, $C_2$ sono due gruppi di nodi e $Cut$ e $Vol$ definiti come segue:

	\begin{equation*}
		Cut(C_1, C_2) = \sum_{i \in C_1, j \in C_2} {w_{ij}}
	\end{equation*}

	\begin{equation*}
		Vol(C) = \sum_{i \in C, j \in V} {w_{ij}}
	\end{equation*}
	dove $Cut(C_1, C_2)$ calcola il peso totale delle connessioni tra $C_1$ e $C_2$, $Vol(C)$ calcola il peso totale degli archi originati dal cluster $C$ e $w_{ij}$ indica il peso dell'arco sotteso tra $i$ e $j$.

	Uno degli algoritmi più importanti appartenente a questa categoria prende il nome di Spectral\cite{spectral}.
	\section{Altri modelli di clutering}
	Come già detto, non esiste un solo modo, o algoritmo che sia, per riuscire a raggruppare degli oggetti. Qualsiasi intuizione potrebbe infatti generare un nuovo modello di pensiero, come successo per il \emph{density-based} e per il \emph{distribution-based}:

	\begin{itemize}
		\item Density-based:
			Negli algoritmi di clustering density-based, il raggruppamento avviene analizzando l'intorno di ogni punto dello spazio, connettendo regioni di punti con densità sufficientemente alta ed eliminando il rumore, ovvero gli elementi appartenenti a regioni con bassa densità\cite{Density_based_clustering}.\\
			L'algoritmo più famoso appartenente a questa categoria è senz'altro DBSCAN.
		
		\item Distribution-based:
			Questa tipologia di algoritmi è quella che si avvicina di più allo studio statistico. I cluster possono essere definiti come insiemi di oggetti che appartengono, probabilmente, alla stessa distribuzione\cite{distribution-based_clustering}. \\
			L'algoritmo che definisce al meglio questa tipologia è il Gaussian Mixture Models.
	\end{itemize}
