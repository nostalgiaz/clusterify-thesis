\chapter{Clustering}

Il \emph{clustering} è una tecnica di raggruppamento di oggetti in insiemi, secondo determinate regole, con l'intento che gli appartenenti allo stesso \emph{cluster} siano più simili di quelli contentuti negli altri gruppi.

Queste regole tendono a minimizzare la \emph{distanza} interna a ciascun gruppo ed a massimizzare quella tra i gruppi stessi: questa distanza viene quantificata per mezzo di misure di similarità [Qui ricordati la differenza tra una distanza e una dissimilarita`].

\section{Clusters e clustering}

Non esiste un solo algoritmo per raggruppare in modo corretto, bensì un solo compito da portare a termine. Questo obiettivo può essere raggiunto in più modi, basandosi su qualsivoglia tipo di nozione col solo fine di raggruppare in modo efficiente e sensato gli oggetti dati. 

Esistono molteplici nozioni a cui ci si può ispirare per l'atto del raggruppamento; la creazione di gruppi caratterizzati da una piccola distanza tra i propri membri, piuttosto che l'utilizzo di particolari distribuzioni statistiche.

Seppur non esista non solo modo per fare clustering, molti di questi algoritmi godono di alcuni tratti in comune che definiscono i cosidetti \emph{modelli}.

I modelli di clustering tipici sono:
\begin{itemize}
	\item Clustering partizionale 
	\item Clustering gerarchico 
	\item Clustering density-based 
\end{itemize}

\subsection{Clustering partizionale}
	% http://en.wikipedia.org/wiki/Cluster_analysis#Centroid-based_clustering
	
	Gli algoritmi di clustering di questa famiglia creano una partizione delle osservazioni minimizzando la seguente funzione di costo:
	\begin{equation*}
	  \sum_{j=1}^{k}E(C_j)
	\end{equation*}
	ove $k$ è il numero dei cluster richiesti in output, $C_j$ è il j-esimo cluster e $E:C \rightarrow R^{+}$ è la funzione di costo associata al singolo cluster.

	Questa tipologia di algoritmi solitamente richiede all'utente di specificare $k$, il numero di cluster distinti che si vogliono raggiundere a processo terminato, e mira ad identificare i gruppi naturali presenti nel dataset, generando una partizione composta da cluster disgiunti la cui unione ritorna il dataset originale.
	
	\subsubsection{Algoritmi conosciuti}
		Gli algoritmi più famosi appartenenti questa categoria sono: 
		\begin{itemize}
		  	\item k-means
		  	\item k-medoids
		  	\item CLARANS
		\end{itemize}

\subsection{Clustering gerarchico}
    % http://en.wikipedia.org/wiki/Cluster_analysis#Connectivity_based_clustering_.28hierarchical_clustering.29

	Gli algoritmi di clustering gerarchico, invece, creano una rappresentazione gerarchia ad albero dei cluster.
	Le strategie per il clustering gerarchico sono tipicamente di due tipi: 
	\begin{itemize}
		\item Agglomerativo
		\item Divisivo
	\end{itemize}
	
	\subsubsection{Metodo agglomerativo}
		Il metodo agglomerativo segue un approccio \emph{bottom up} al problema dove, inzialmente, si ha un cluster per ogni oggetto e, successivamente, si procede all'unione di questi cluster, basando la selezione dei cluster da unire ad una \emph{funzione di similarità}.

	\subsubsection{Metodo divisivo}
		Il metodo divisivo, invece, segue un approccio \emph{top down} al problema dove, inizialemtente, si ha un unico cluster contente tutti gli oggetti e, via via, viene suddiviso in più sotto-cluster, basando la selezione del cluster da dividere ad una \emph{funzione di similarità}. Solitamente si impone un numero minimo di elementi che ogni cluster deve contenere alla fine del processo. 

	\subsubsection{Dissimilarità tra cluster}
		Nella maggior parte dei metodi di clustering gerarchico si fa uso di metriche specifiche che quantificano la distanza tra coppie di elementi e di un criterio di collegamento che specifica la dissimilarità di due insiemi di elementi (cluster) come funzione della distanza a coppie tra elementi nei due insiemi.

	\subsubsection{Metriche}
		La scelta della metrica influenza la forma dei cluster, poiché alcuni elementi possono essere più vicini utilizzando una data distanza e più lontani utilizzandone un'altra.
		Le metriche comuni sono le seguenti:
		\begin{itemize}
		  	\item Distanza euclidea
		  	\item Distanza di Manhattan
		\end{itemize}

	\subsubsection{Criteri di collegamento}
		Il criterio di collegamento specifica la distanza tra insiemi di elementi come funzione di distanze tra gli elementi negli insiemi.
		I criteri di collegamento comuni sono i seguenti:
		\begin{itemize}
			\item Complete linkage: calcola la distanza tra i due cluster come la distanza massima tra elementi appartenenti ai due clusters
			\item Minimum o single-linkage: calcola la distanza tra i due cluster come la distanza minima tra elementi appartenenti a cluster diversi
			\item Average linkage: calcola la distanza tra i due cluster come la media delle distanze tra i singoli elementi
		\end{itemize}

	\subsubsection{Algoritmi conosciuti}		
		Gli algoritmi più famosi appartenenti questa categoria sono:
		\begin{itemize}
			\item SLINK \emph{(single-linkage)}
		  	\item CLINK \emph{(complete-linkage)}
		\end{itemize}

\subsection{Clustering density-based}
	Negli algoritmi di clustering density-based il raggruppamento avviene analizzando l'intorno di ogni punto dello spazio. In particolare, viene considerata la densità di punti in un intorno di raggio fissato.
	
	\subsubsection{Algoritmi conosciuti}		
		Gli algoritmi più famosi appartenenti questa categoria sono:
		\begin{itemize}
		  	\item Bdscan
		\end{itemize}