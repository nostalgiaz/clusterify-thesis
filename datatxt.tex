\subsection{dataTXT}
% https://www.mashape.com/dandelion/datatxt-1#!documentation
	DataTXT è un insieme di API semantiche sviluppate da SpazioDati S.r.l.  \url{http://spaziodati.eu} che mira a estrarre significato da testi scritti in diverse lingue. Questo prodotto si diversifica dagli altri dello stesso genere in quanto è stato pensato e ottimizzato per lavorare su testi molto corti, come i tweet.

	DataTXT può estrarre da un testo delle \emph{entità}, può categorizzare documenti in categorie stabilite dall'utente stesso e molto altro.

	%\subsubsection{Grafo di dataTXT}
%		DataTXT per poter eseguire i suoi algoritmi ha bisogno di basarsi su di un grafo pre-calcolato; questo ha $n+r$ nodi, dove $n$ sono gli snippet $S =  \{S_1, ..., S_n\}$ e $r$ sono i topic $T = \{t_1, ..., t_r\}$. 
%
%		Dato uno snippet $s$ e un topic $t$, denotiamo come $p(s,t)$ lo score che dataTXT assegna all'annotazione di $s$ con il topic $t$. Questo valore rappresenta l'importanza che un topic $t$ ha su di uno snippet $s$ e viene utilizzato come valore per l'arco da va da $s$ a $t$ nel grafo stesso.
%
%	Dati due topic $t_a$ e $t_b$, per esempio due pagine di Wikipedia, si può misurare la loro \emph{relatedness} $rel(t_a, t_b)$ utilizzano la seguente funzione che si basta sul numero di citazioni e co-citazioni delle due pagine di Wikipedia:
%	\begin{equation*}
%	  rel(t_a, t_b) = \frac{log(| in(t_a) |) - log(| in(t_a) \cap in(t_b)|)}{log(W) - log(|in(t_b)|)}
%	\end{equation*}
%	dove $in(t)$ è il set di archi entranti nella pagina $t$ e $W$ è il numero totale delle pagine di Wikipedia\cite{datatxt_graph}.

	\subsubsection{dataTXT NEX}
		DataTXT NEX (\emph{Named Entity eXtraction}) è un API che permette di localizzazione e di estratte da un testo entità quali persone, organizzazioni, luoghi, espressioni di tempo, ecc.

		Per ottenere queste informazioni basta richiedere via HTTP l'elaborazione dei dati  prefissati in input. Segue un esempio: \url{https://api.dandelion.eu/datatxt/nex/v1/?lang=en&text=The%20doctor%20says%20an%20apple%20is%20better%20than%20an%20orange&include=type,abstract,categories,lod&$app_id=YOUR_APP_ID&$app_key=YOUR_APP_KEY}

		Dopo l'elaborazione di una qualsiasi richiesta, una risposta viene consegnata.

		Come  tutte le chiamate HTTP, anche le risposte di dataTXT sono composte da un \emph{header} e da un \emph{body}.

		Nella prima sezione sono presenti molte informazioni utili quali il numero di crediti utilizzati, quelli rimanente e la data in cui questi torneranno al valore iniziale.
		
		\begin{lstlisting}
Connection: keep-alive
Content-Length: 2748
Content-Type: application/json;charset=UTF-8
Date: Wed, 21 Oct 2015 16:29:37 GMT
Server: Apache-Coyote/1.1
X-DL-units: 1
X-DL-units-left: 999
X-DL-units-reset: 2015-10-22 00:00:00 +0000
		\end{lstlisting}

		Nella seconda, invece, sono presenti i dati veri e propri. \emph{Annotations} è la lista composta dagli elementi trovati, con vari dettagli; questi elementi sono filtrati da alcuni parametri che possono essere passati nella richiesta,  ad esempio la \emph{min\_confidence}.

		\begin{lstlisting}
{
  "timestamp": "2015-10-21T16:29:37",
  "time": 2,
  "lang": "en",
  "annotations": [{
      "...": "...",
  }, {
    "abstract": "The apple is the fruit of the apple tree...",
    "id": 18978754,
    "title": "Apple",
    "start": 19,
    "categories": [
      "Apples",
      "Malus",
      "Plants described in 1803",
      "Sequenced genomes"
    ],
    "lod": {
      "wikipedia": "http://en.wikipedia.org/wiki/Apple",
      "dbpedia": "http://dbpedia.org/resource/Apple"
    },
    "label": "Apple",
    "types": [
      "http://dbpedia.org/ontology/Eukaryote",
      "http://dbpedia.org/ontology/Plant",
      "http://dbpedia.org/ontology/Species"
    ],
    "confidence": 0.7869,
    "uri": "http://en.wikipedia.org/wiki/Apple",
    "end": 24,
    "spot": "apple"
  }, {
    "...": "...",
  }]
}
		\end{lstlisting}

	\subsubsection{dataTXT REL}
		DataTXT REL  è un servizio che, ad oggi, non è ancora presente sul mercato, ma che permetterà di capire quando un topic $t_a$ è correlato ad un topic $t_b$, ricavando questo valore dalla seguente formula:
		\begin{equation*}
			rel(t_a, t_b) = \frac{log(| in(t_a) |) - log(| in(t_a) \cap in(t_b)|)}{log(W) - log(|in(t_b)|)}
		\end{equation*}

		Come dataTXT-NEX, anche dataTXT-REL sarà interrogabile nello stesso modo: \url{https://api.dandelion.eu/datatxt/rel/v1/?topic1=Roma&topic2=Trento&lang=it&$app_id=YOUR_APP_ID&$app_key=YOUR_APP_KEY}.

		L'\emph{header} della risposta sarà uguale a quello di dataTXT-NEX, mentre il \emph{body} sarà strutturato come segue:

		\begin{lstlisting}
{
  "time": 2,
  "relatedness": [{
    "weight": 0.7127059,
      "topic1": {
        "input": "Roma",
        "topic": {
          "id": 1209297,
          'title": "Roma",
          "uri": "http://it.wikipedia.org/wiki/Roma",
          "labe"l: "Roma"
        }
      },
      "topic2": {
        "input": "Trento",
        "topic": {
          "id": 100137,
          "title": "Trento",
          "uri": "http://it.wikipedia.org/wiki/Trento",
          "label": "Trento"
        }
      }
  }],
  "lang": "it",
  "timestamp": "2014-05-14T21:41:58.171"
}
		\end{lstlisting}


