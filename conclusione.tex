\chap{Conclusioni e sviluppi futuri}
	Clusterify è un prodotto pronto all'uso e già utilizzato da SpazioDati per pubblicizzare  uno dei molteplici casi in cui dataTXT può essere utilizzato. Nato dall'idea di mostrare come la forza del clustering può essere utilizzata anche su testi molto corti, è stata soggetto di molti studi nel team di SpazioDati per poter creare un ulteriore prodotto vendibile al pubblico pagante.

	Clusterify nasce però come \emph{proof of concept} per mostrare queste potenzialità. In futuro potrebbe essere esteso con l'implementazione di un \emph{workflow} totalmente dinamico, dove il processo di caricamento, annotazione e clustering dei testi possa essere di tipo incrementale. In questo scenario l'utente potrebbe abbandonare l'interfaccia tipica offerta da Twitter per avere il proprio stream aggiornato in tempo reale su Clusterify.

	Questa composizione presenta tutti gli aspetti legati al \emph{core} di Clusterify, i suoi possibili miglioramenti e le sue limitazioni. Si può anche affermare che è comunque un buon punto di partenza per nuovi programmatori che vorrebbero contribuire al progetto, portando nuove idee e nuova forza.

	Penso che Clusterify sia uno strumento ben pensato per essere successivamente esteso e migliorato e che sia molto importante per i futuri clienti di SpazioDati; spero, infine, che questo progetto riesca ancora a svilupparsi.