\section{Altri modelli di clutering}
	Come già detto, non esiste un solo modo, o algoritmo che sia, per riuscire a raggruppare degli oggetti. Qualsiasi intuizione potrebbe infatti generare un nuovo modello di pensiero, come successo per il \emph{density-based} e per il \emph{distribution-based}:

	\begin{itemize}
		\item Density-based:
			Negli algoritmi di clustering density-based, il raggruppamento avviene analizzando l'intorno di ogni punto dello spazio, connettendo regioni di punti con densità sufficientemente alta ed eliminando il rumore, ovvero gli elementi appartenenti a regioni con bassa densità\cite{Density_based_clustering}.\\
			L'algoritmo più famoso appartenente a questa categoria è senz'altro DBSCAN.
		
		\item Distribution-based:
			Questa tipologia di algoritmi è quella che si avvicina di più allo studio statistico. I cluster possono essere definiti come insiemi di oggetti che appartengono, probabilmente, alla stessa distribuzione\cite{distribution-based_clustering}. \\
			L'algoritmo che definisce al meglio questa tipologia è il Gaussian Mixture Models.
	\end{itemize}