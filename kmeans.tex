K-means è il nome di un algoritmo di clustering che punta a creare $k$  gruppi distinti di uguale varianza, minimizzando l'inerzia del gruppo. Questo algoritmo richiede che il numero di cluster sia definito a priori. 

L'algoritmo punta a scegliere $k$ centroidi $C$ che minimizzano lo scarto quadratico medio con un dataset $X$ di $n$ elementi $(x_1,..., x_n)$, creando $k$ insiemi $S = {S_1, ..., S_K}$:

\begin{equation*}
	\underset {\mathbf{S}} {\operatorname{arg\,min}}  \sum_{i=1}^{k} \sum_{\mathbf x_j \in S_i} || x_j - \mu_i ||^2 
\end{equation*}

L'algoritmo in se è composto da tre passi:
\begin{enumerate}
	\item Scelta dei centroidi iniziali;
	\item Assegnamento di ogni componente;
	\item Creazione di nuovi centroidi;
\end{enumerate}

Dopo aver scelto i primi $k$ centroidi dalle componenti del dataset $X$, l'algoritmo continua ad eseguire due operazioni finché i centroidi proposti possono essere considerati corretti, ovvero che la propria posizione non venga modificata con il passare delle iterazioni. 

L'assegnamento di ogni componente ad un centroide si esegue utilizzando la distanza euclidea al quadrato, questa sarà assegnata al punto medio più vicino; si può calcolare la distribuzione delle componenti in questo modo:
\begin{equation*}
	S_i^{(t)} = \big \{ x_p : || x_p - m^{(t)}_i ||^2 \le || x_p - m^{(t)}_j ||^2 \ \forall j, 1 \le j \le k \big\}
\end{equation*}
dove ogni $x_p$ è assegnato a esattamente un $S^{(t)}$, anche se idealmente potrebbe essere assegnato a più di un centroide.

La creazione dei nuovi centroidi utilizza, invece, il valore medio di ogni componente appartenente ad ogni centroide precedentemente definito.
\begin{equation*}
	m^{(t+1)}_i = \frac{1}{|S^{(t)}_i|} \sum_{x_j \in S^{(t)}_i} x_j 
\end{equation*}

Per quanto riguarda la scelta dei centroidi ci si può basare sul caso, scegliendo in modo \emph{random} i punti, con l'idea che con il passare delle iterazioni queste stime si sistemeranno; la seconda scelta vede il posizionamento dei centroidi in modo equidistante dagli tra loro, col fine di cercare di massimizzare l'area coperta e minimizzare il numero di iterazioni necessarie per giungere nella situazione finale di stallo.