\section{Clustering gerarchico}
	Gli algoritmi di clustering gerarchico creano una rappresentazione gerarchia ad albero dei cluster.
	Le strategie sono tipicamente di due tipi: 
	\begin{itemize}
		\item Metodo agglomerativo \emph{(bottom-up)} \\
		Si inizia col creare ed associare un cluster ad ogni oggetto, e si procede poi con l'unione di questi, basando la selezione degli insiemi da unire su una \emph{funzione di similarità}.

		\item Metodo divisivo \emph{(top-down)} \\
		Partendo da un unico cluster contenente tutti gli oggetti, lo si divide basando la scelta della selezione dell’insieme da dividere su una \emph{funzione di similarità}.
	\end{itemize}

	Questa tipologia di algoritmi necessita anche di alcuni criteri di collegamento che specificano la dissimilarità di due insiemi utilizzando la distanza calcolabile tra gli stessi. Questi criteri possono essere: 
	\begin{itemize}
		\item Complete linkage: distanza massima tra elementi appartenenti a due cluster
			\begin{equation*}
				\max \, \{\, d(a,b) : a \in C_1,\, b \in C_2 \,\}
			\end{equation*}
		\item Minimum o single-linkage: distanza minima tra elementi appartenenti a cluster diversi
			\begin{equation*}
			 	\min \, \{\, d(a,b) : a \in C_1,\, b \in C_2 \,\}
			\end{equation*}

		\item Average linkage: la media delle distanze dei singoli elementi
			\begin{equation*}
				\frac{1}{|C_1| |C_2|} \sum_{a \in C_1 }\sum_{ b \in C_2} d(a,b). 
			\end{equation*}
	\end{itemize}
	dove $C_1$ e $C_2$ rappresentano i due cluster da unire e $d$ la distanza prescelta.

	Gli algoritmi più importanti appartenenti questa categoria sono SLINK (single-linkage) e CLINK (complete-linkage)\cite{clustering_gerarchico}.
