\section{Clustering distribution-based}
	Questa tipologia di algoritmi è quella che si avvicina di più allo studio statistico. I cluster possono essere definiti come l'insieme di oggetti che appartengono, probabilmente, alla stessa distribuzione\cite{distribution-based_clustering}.

	Anche  se dal punto di vista teorico questi modelli sono eccellenti, soffrono di un problema chiave chiamato \emph{overfitting}. 

	Si parla di overfitting quando nell'apprendimento automatico e nel data mining quando un algoritmo che prevede apprendimento viene allenato usando un training set e si assume che quest'algoritmo raggiungerà uno stato in cui sarà in grado di prevedere gli output per tutti gli altri esempi. Tuttavia, è possibile che a causa di un apprendimento svolto in modo sbagliato, il modello potrebbe adattarsi a caratteristiche che sono specifiche del training set ma che non hanno riscontro nel resto dei casi\cite{overfitting}.

	L'algoritmo che definisce al meglio questa tipologia è il Gaussian Mixture Models.
