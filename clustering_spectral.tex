\section{Clustering spectral}
	Questa tipologia nasce dal bisogno di eliminare delle mancanze che colpiscono gli altri metodi. 

	I metodi partizionali riescono a creare cluster solamente di forma sferica, basandosi, come abbiamo già visto, su un concetto di centro; al contrario i metodi density-based hanno un approccio troppo sensibile ai parametri dati.

	Per risolvere questi problemi si utilizza un grafo di similarità con vertici le componenti da clusterizzare e, come valore degli archi, la similarità delle componenti tra cui è sotteso.

	Una volta creato il grafo si procede con una serie di tagli minimi, che possono essere trovati in questo modo:

	\begin{equation*}
		NormCut(C_1, C_2) =  \frac{Cut(C_1, C_2)}{Vol(C_1)} + \frac{Cut(C_1, C_2)}{Vol(C_2)}
	\end{equation*}
	dove $C_1$, $C_2$ sono due gruppi di nodi,  $w_{ij}$ il peso dell'arco sotteso tra $i$ e $j$, $Cut$ e $Vol$ definito come segue:

	\begin{equation*}
		Cut(C_1, C_2) = \sum_{i \in C_1, j \in C_2} {w_{ij}}
	\end{equation*}

	\begin{equation*}
		Vol(C) = \sum_{i \in C, j \in V} {w_{ij}}
	\end{equation*}

	L'algoritmo più famoso appartenente a questa categoria è senz'altro Spectral.