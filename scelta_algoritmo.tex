Una volta definiti i quattro algoritmi è necessario capire quale performa meglio nel clusterizzare argomenti secondo la relatedness offerta da dataTXT.

Partendo da un'analisi preliminare, si capisce che l'algoritmo che potrebbe ottenere risultati migliori rispetto agli altri è l'Affinity propagation in quanto è l'unico che non richiede in input il numero di cluster attesi: questo elemento è fondamentale perché è impossibile valutare a priori quanti cluster ogni dataset potrebbe generare. 
Affinity propagation è stato applicato su concetti molto vari, dall'analisi di una partita di basket all'individuazione di punti interessanti in un'immagine, e per questo si stima che potrebbe rispondere in modo ottimale al problema essendo molto generico.

È stato chiesto a dieci persone di creare dei cluster partendo dalle entità estratte dagli ultimi 50 tweet che popolavano il loro newsfeed.

La scelta del campione non è stata casuale: si sono cercate delle persone che avessero zone di interesse o vaste o molto specifiche, e al contempo si è mirato a coprire molti ambiti, dalla musica alla politica. Questa scelta ha portato a valutare molti aspetti degli algoritmi, dal clusterizzare argomenti molto correlati tra loro al caso opposto. 

Le persone così scelte possono essere rappresentate in questo modo:
\begin{itemize}
	\item MartinBrugrara: giovane studente di informatica con la passione dei database, utilizza Twitter per tenersi aggiornato circa le nuove \emph{release} dei progetti che segue;
	\item AndreaDellera: musicista e cantante di un nuovo gruppo, utilizza questo social network per seguire e per essere seguito dai propri fan;
	\item gi4n1uc4: ragazzo parigino amante del paese in cui è nato a tal punto da tenersi informato degli eventi che accadono;
	\item apheniti; ragazza romana amante delle serie TV e del cinema in generale, utilizzaTwitter per rimanere in contatto con i suoi attori preferiti;
	\item john\_frigo: ragazzo con la passione dei videogiochi, in tutti i sensi. Sogna di diventare un programmatore per qualche company internazionale;
	\item edorigatti: ragazzo con la passione per le macchine e per le moto, al contempo intraprende studi in materie scientifiche;
	\item cosaunlama: appassionato di cinema, utilizza Twitter per sapere prima degli altri le nomination agli Oscar;
	\item michelaelle: giovane fotografa interessata di belle arti, utilizza questo social network per seguire i suoi artisti preferiti e per venire a conoscenza di una mostra a cui vorrebbe andare; 
	\item SpiritualGuru: ragazzo con la passione dell'informatica legata al design, segue persone che danno lui ispirazione;
	\item carlotta\_93: ragazza con la passione della politica e dell'attualità, utilizza Twitter per rimanere aggiornata sulle decisioni che prendono i vari partiti.
\end{itemize}

Il prodotto di queste interviste è stato paragonato con l'output di tutti gli algoritmi proposti. 

Per questo compito è stata utilizzata la funzione ``adjusted\_rand\_score'' offerta da scikit-learn -- \url{http://scikit-learn.org/stable/modules/generated/sklearn.metrics.adjusted_rand_score.html} -- che permette di calcolare la similarità tra i cluster proposti e quello risultante. Per fare questo si considerano tutte le coppie di elementi e si contano quelle che sono state assegnate allo stesso o ad un altro cluster.

Per permettere agli intervistati di formare questi cluster è stato creato un apposito file su google drive -- \url{https://docs.google.com/spreadsheets/d/1RNte6_I-cO5A23dKoaAyIRX5iWOGAEN3S25VIpXQLWU/edit?usp=sharing} -- con gli argomenti da clusterizzare pre-caricati ed è stato chiesto loro di segnare con lo stesso numero quelli che secondo loro avrebbero dovuto appartenere allo stesso cluster. 

Lanciando \texttt{python manage.py analyze} dalla \emph{root} di Clusterify questi dati vengono analizzati. I risultati sono riportati nel grafico:

\catcode`\_=12
\begin{tikzpicture}
	\begin{axis} [
   	 	ybar,
		width=12cm,
	        	height=8cm,
        		ymin=0,
	  	ymax=150,  
		bar width=3pt,
        		legend columns=2,
    		legend style={draw=none,nodes={inner sep=3pt}},,
    		ylabel={\% successo},
    		symbolic x coords={
    			MartinBrugrara,
    			AndreaDellera,
    			gi4n1uc4,
    			apheniti,
    			john_frigo,
		    	edorigatti,
			cosaunlama,
			michelaelle,
			SpiritualGuru,
			carlotta_93
    		},
    		xtick=data,
    		nodes near coords,
    		nodes near coords align={vertical},
		nodes near coords={
			\pgfmathprintnumber[fixed zerofill, precision=2]{\pgfplotspointmeta}
		},
		every node near coord/.append style={
			font=\tiny,
			anchor=west,
                		rotate=90
        		},
		x tick label style={rotate=45, anchor=east},
    	]
		\addplot coordinates {
			(MartinBrugrara, 89.61) 
			(AndreaDellera, 71.72) 
			(gi4n1uc4, 85.81)
			(apheniti, 77.02)
			(john_frigo, 94.47)
			(edorigatti, 81.56)
			(cosaunlama, 79.00)
			(michelaelle, 76.43)
			(SpiritualGuru, 81.04)
			(carlotta_93, 74.23)
		};
		\addplot coordinates {
			(MartinBrugrara, 57.08) 
			(AndreaDellera, 55.56) 
			(gi4n1uc4, 56.89)
			(apheniti, 64.60)
			(john_frigo, 59.87)
			(edorigatti, 57.96)
			(cosaunlama, 58.86)
			(michelaelle,  57.15)
			(SpiritualGuru, 56.46)
			(carlotta_93, 65.06)
		};
		\addplot coordinates {
			(MartinBrugrara, 64.26) 
			(AndreaDellera, 67.84) 
			(gi4n1uc4, 65.23)
			(apheniti, 64.77)
			(john_frigo, 61.94)
			(edorigatti, 61.73)
			(cosaunlama, 67.00)
			(michelaelle,  64.50)
			(SpiritualGuru, 62.50)
			(carlotta_93, 71.69)
		};
		\addplot coordinates {
			(MartinBrugrara, 66.82) 
			(AndreaDellera, 63.63) 
			(gi4n1uc4, 64.32)
			(apheniti, 64.44)
			(john_frigo, 63.59)
			(edorigatti, 62.81)
			(cosaunlama, 67.00)
			(michelaelle,  64.33)
			(SpiritualGuru, 62.78)
			(carlotta_93, 72.88)
		};
		\legend{Affinity Propagation, Specral, K-means, Hierarchical}
	\end{axis}
\end{tikzpicture}
\catcode`\_=8

Come si può dedurre dal grafico, ogni persona che ha creato il proprio cluster ideale si è avvicinata inconsciamente alla soluzione proposta dall'\emph{Affinity Propagation}. 

Una volta raccolti i dati è stato mostrato al campione intervistato il risultato proposto su di un nuovo dataset, generato sempre dall'account Twitter dei diretti interessati, ed il risultato nel 90\% dei casi, era sensato ed utilizzabile.

Alla luce di questi risultati, Clusterify utilizzerà Affinity Propagation.
