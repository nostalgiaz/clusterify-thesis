\chap{Prefazione}

\epigraph{``È evidente che l’uomo sia un essere sociale più di ogni ape e più di ogni animale da gregge. Infatti, la natura non fa nulla, come diciamo, senza uno scopo: l’uomo è l’unico degli esseri viventi a possedere la parola."}{--- \textup{Aristotele}, Politica }

L'uomo, in quanto essere sociale, è portato al bisogno di comunicare. Dall'invenzione del telegrafo all'utilizzo di Internet è cambiato solamente il mezzo di divulgazione del pensiero ma non l'atto di volerlo condividere con il resto dell'umanità.

I pensieri vengono talvolta espressi sotto forma di testi che un individuo scrive nella speranza che vengano letti da altri. La grande mole di messaggi può essere elaborata per ricavarne informazioni utili, che potenzialmente potrebbero risolvere alcuni problemi.

Clusterify ha l'intento di mostrare la potenzialità del clustering non supervisionato, processo in grado di aprire le porte a molteplici risultati: dalla possibilità di filtrare testi per una macro-area interessata, al profilare persone interessate ad un'attività per cercare di anticipare il mercato proponendo loro quello che vogliono prima di sapere ciò di cui hanno bisogno.

 I settori nei quali si può utilizzare il clustering sono davvero illimitati in quanto nessuna restrizione viene imposta bensì viene permesso ai dati stessi di adattare l'ambiente, rendendo così questa tecnica economica e vincente.