\chap{Prefazione}

\epigraph{``È evidente che l’uomo sia un essere sociale più di ogni ape e più di ogni animale da gregge. Infatti, la natura non fa nulla, come diciamo, senza uno scopo: l’uomo è l’unico degli esseri viventi a possedere la parola."}{--- \textup{Aristotele}, Politica }

L'uomo, in quanto essere sociale, per natura è portato al bisogno di comunicare. Dall'invenzione del telegrafo all'utilizzo di Internet è cambiato solamente il mezzo di divulgazione, ma non l'atto di voler condividere pensieri con il resto dell'umanità.

Questi pensieri vengono talvolta espressi sotto forma di testi, di dimensione variabile, che un individuo scrive nella speranza che da altri vengano letti. La grande mole di messaggi può essere elaborata per ricavarne informazioni utili, potenzialmente per la risoluzione di alcuni problemi.

Clusterify è un'applicazione web che mira alla possibilità di risolvere più problemi contemporaneamente: dividere un ammasso di  in insiemi distinti, dove ognuno è caratterizzato da un forte legame che le proprie componenti hanno tra di loro, e da un legame debole nei confronti delle altre. Questo processo può aprire le porte a molteplici risultati, come la possibilità di filtrare testi per una macro-area interessata, piuttosto che caratterizzare persone interessate ad un'attività per capire come meglio investire nel prossimo semestre.

Il mio progetto ha l'intento di mostrare la potenzialità del clustering non supervisionato per la suddivisione di testi. Nella fattispecie si ha intenzione di applicare degli algoritmi di clustering per definire questi sottogruppi ed, una volta composti, utilizzarli come strumento per filtrare gli stessi messaggi. Supponendo che l'utente medio di Twitter non parli solamente dell'argomento per cui è stato seguito, è possibile che questo scriva di materie al nostro utente interessanti o, al contrario, ininfluenti. 

Clusterify, così facendo, permette di leggere solamente  per lui interessanti. 