\chapter{Clusterify}
	Clusterify è un'applicazione web che offre agli utenti la possibilità di leggere gli stessi tweet che leggerebbero sul \emph{newsfeed} dello stesso Twitter, suddivisi però in insiemi definiti dinamicamente in modo da permettere loro di soffermarsi solamente su testi appartenenti alle categorie che questi reputano importanti. L'idea di quest'applicazione è nata dal bisogno di voler leggere solo in parte i tweet che una data persona pubblica. 

	Supponiamo che un utente $X$, aspirante cuoco, segua $Y$, cuoca di grande importanza e neo-mamma, e che quest'ultima pubblichi questi due tweet:
	\begin{inparaenum}[\itshape a\upshape)]
		\item \#Plumcake alla \#Nutella: continuano gli esperimenti @GialloZafferano ;)
		\item Giocare una battaglia a \#palledineve e perdere con il proprio piccolo di 3 anni <3
	\end{inparaenum}.
	$X$ probabilmente sarà molto interessato alla ricetta del \emph{Plumcake} ma, al contempo, non attirato dalla vita personale di $Y$.

	Clusterify dividerà questi due tweet in due gruppi distinti e, così, $X$ avrà la possibilità di leggere tutti i tweet riferiti ad ambiti culinari in un unico gruppo senza essere disturbato da storie di vita personale, recensioni di film e quant'altro.

\section{Background}
	\subsection{Twitter API}
	Twitter Inc. è un social network che permette ai propri utenti di scrivere e di leggere \emph{micropost}, messaggi lunghi al più 140 caratteri, comunemente chiamati \emph{tweet}. 

	Questa piattaforma, creata nel 2006 da Jack Dorsey, Evan Williams, Biz Stone e Noah Glass, ha spopolato fino a raggiungere nel 2013 più di 200 milioni di utenti attivi e più di 400 milioni di visite al giorno\cite{twitter_data}.

	Ad oggi, Twitter è l'undicesimo sito internet visitato quotidianamente, secondo nei social network, preceduto solamente da Facebook\cite{twitter_alexa}.

	Twitter possiede, come la maggior parte dei social network, un servizio interrogabile via API per ottenere e per caricare dati: nel primo caso si possono richiedere i tweet che compongono il \emph{newsfeed} di un dato utente, i messaggi privati che due utenti si scambiano ed ancora alcuni dati relativi al profilo, come nome e cognome, nickname, immagine utente e così via; nel secondo caso, invece, si può pubblicare un tweet come se l'utente lo facesse dall'interfaccia web oppure modificare dati che Twitter ha salvato nel proprio database. 

	In entrambi i casi serve che l'utente interessato ``firmi'' un consenso che permetterà al programmatore di interagire con questi dati. 

	\subsubsection{REST API v1.1}
		Il modo che Twitter offre per interagire con i propri dati è la REST API. 
	
		Con REST (Representational State Transfer) si indica un tipo di architettura software basato sull'idea di utilizzare la comunicazione tra macchine per mezzo di richieste HTTP.

		Le applicazioni basate su questo tipo di architettura vengono chiamate RESTful ed utilizzano, appunto, HTTP per tutte le operazioni di \emph{CRUD}: Create, Read, Update e Remove.

		Twitter mette a disposizione, nella documentazione per i programmatori\cite{twitter_doc}, una lunga lista di possibili richieste -- \url{https://dev.twitter.com/docs/api/1.1} -- e, per ognuna di queste, una pagina dedicata dove vengono elencati i dati richiesti in input, i possibili filtri applicabili, la struttura dell'output che questa richiesta genererà ed infine un esempio richiesta/risposta per chiarire le idee.

		% https://dev.twitter.com/docs/auth
		OAuth, il protocollo di autenticazione che Twitter utilizza\cite{twitter_auth}, permette agli utenti di approvare che un'applicazione agisca al loro posto, senza il bisogno di condividere i dati sensibili, quali username e password. OAuth scambia dei token per evitare il passaggio di questa tipologia di dati.

		Gli sviluppatori possono così pubblicare e interagire con dati protetti, come ad esempio i service provider e al contempo proteggono le credenziali dei loro utenti\cite{twitter_auth_faq}.

% PARLARE DI UNA POSSIBILE RICHIESTA!
	\subsection{dataTXT}
% https://www.mashape.com/dandelion/datatxt-1#!documentation
	DataTXT è un insieme di API semantiche sviluppate da SpazioDati S.r.l.  -- \url{http://spaziodati.eu} -- che mira a estrarre significato da testi scritti in diverse lingue. Questo prodotto si diversifica dagli altri dello stesso genere in quanto è stato pensato e ottimizzato per lavorare su testi molto corti, come i tweet.

	DataTXT può estrarre da un testo delle \emph{entità}, può categorizzare documenti in categorie stabilite dall'utente stesso e molto altro.

	%\subsubsection{Grafo di dataTXT}
%		DataTXT per poter eseguire i suoi algoritmi ha bisogno di basarsi su di un grafo pre-calcolato; questo ha $n+r$ nodi, dove $n$ sono gli snippet $S =  \{S_1, ..., S_n\}$ e $r$ sono i topic $T = \{t_1, ..., t_r\}$. 
%
%		Dato uno snippet $s$ e un topic $t$, denotiamo come $p(s,t)$ lo score che dataTXT assegna all'annotazione di $s$ con il topic $t$. Questo valore rappresenta l'importanza che un topic $t$ ha su di uno snippet $s$ e viene utilizzato come valore per l'arco da va da $s$ a $t$ nel grafo stesso.
%
%	Dati due topic $t_a$ e $t_b$, per esempio due pagine di Wikipedia, si può misurare la loro \emph{relatedness} $rel(t_a, t_b)$ utilizzano la seguente funzione che si basta sul numero di citazioni e co-citazioni delle due pagine di Wikipedia:
%	\begin{equation*}
%	  rel(t_a, t_b) = \frac{log(| in(t_a) |) - log(| in(t_a) \cap in(t_b)|)}{log(W) - log(|in(t_b)|)}
%	\end{equation*}
%	dove $in(t)$ è il set di archi entranti nella pagina $t$ e $W$ è il numero totale delle pagine di Wikipedia\cite{datatxt_graph}.

	\subsubsection{dataTXT NEX}
		DataTXT NEX (\emph{Named Entity eXtraction}) è un API che permette di localizzare e di estrarre da un testo entità quali persone, organizzazioni, luoghi, espressioni di tempo, ecc.

		Per ottenere queste informazioni basta richiedere via HTTP l'elaborazione dei dati  prefissati in input. Segue un esempio: \url{https://api.dandelion.eu/datatxt/nex/v1/?lang=en&text=The%20doctor%20says%20an%20apple%20is%20better%20than%20an%20orange&include=type,abstract,categories,lod&$app_id=YOUR_APP_ID&$app_key=YOUR_APP_KEY}

		Dopo l'elaborazione di una qualsiasi richiesta, una risposta viene consegnata.

		Come  tutte le chiamate HTTP, anche le risposte di dataTXT sono composte da un \emph{header} e da un \emph{body}.

		Nella prima sezione sono presenti molte informazioni utili quali il numero di crediti utilizzati, quelli rimanenti e la data in cui questi torneranno al valore iniziale, come di seguito mostrato:
		
		\begin{lstlisting}
Connection: keep-alive
Content-Length: 2748
Content-Type: application/json;charset=UTF-8
Date: Wed, 21 Oct 2015 16:29:37 GMT
Server: Apache-Coyote/1.1
X-DL-units: 1
X-DL-units-left: 999
X-DL-units-reset: 2015-10-22 00:00:00 +0000
		\end{lstlisting}

		Nella seconda, invece, sono presenti i dati veri e propri. \emph{Annotations} è la lista composta dagli elementi trovati, con vari dettagli; questi elementi sono filtrati da alcuni parametri che possono essere passati nella richiesta,  ad esempio la \emph{min\_confidence}.

		\begin{lstlisting}
{
  "timestamp": "2015-10-21T16:29:37",
  "time": 2,
  "lang": "en",
  "annotations": [{
      "...": "...",
  }, {
    "abstract": "The apple is the fruit of the apple tree...",
    "id": 18978754,
    "title": "Apple",
    "start": 19,
    "categories": [
      "Apples",
      "Malus",
      "Plants described in 1803",
      "Sequenced genomes"
    ],
    "lod": {
      "wikipedia": "http://en.wikipedia.org/wiki/Apple",
      "dbpedia": "http://dbpedia.org/resource/Apple"
    },
    "label": "Apple",
    "types": [
      "http://dbpedia.org/ontology/Eukaryote",
      "http://dbpedia.org/ontology/Plant",
      "http://dbpedia.org/ontology/Species"
    ],
    "confidence": 0.7869,
    "uri": "http://en.wikipedia.org/wiki/Apple",
    "end": 24,
    "spot": "apple"
  }, {
    "...": "...",
  }]
}
		\end{lstlisting}

	\subsubsection{dataTXT REL}
		DataTXT REL  è un servizio che, ad oggi, non è ancora presente sul mercato ma che permetterà di capire quando un topic $t_a$ è correlato ad un topic $t_b$, ricavando questo valore dalla seguente formula:
		\begin{equation*}
			rel(t_a, t_b) = \frac{log(| in(t_a) |) - log(| in(t_a) \cap in(t_b)|)}{log(W) - log(|in(t_b)|)}
		\end{equation*}
		dove $in(t)$ è il set di archi entranti nella pagina $t$ e $W$ è il numero totale delle pagine di Wikipedia\cite{datatxt_graph}.

		Come dataTXT-NEX, anche dataTXT-REL sarà interrogabile nello stesso modo: \url{https://api.dandelion.eu/datatxt/rel/v1/?topic1=Roma&topic2=Trento&lang=it&$app_id=YOUR_APP_ID&$app_key=YOUR_APP_KEY}.

		L'\emph{header} della risposta sarà uguale a quello di dataTXT-NEX, mentre il \emph{body} sarà strutturato come segue:

		\begin{lstlisting}
{
  "time": 2,
  "relatedness": [{
    "weight": 0.7127059,
      "topic1": {
        "input": "Roma",
        "topic": {
          "id": 1209297,
          'title": "Roma",
          "uri": "http://it.wikipedia.org/wiki/Roma",
          "labe"l: "Roma"
        }
      },
      "topic2": {
        "input": "Trento",
        "topic": {
          "id": 100137,
          "title": "Trento",
          "uri": "http://it.wikipedia.org/wiki/Trento",
          "label": "Trento"
        }
      }
  }],
  "lang": "it",
  "timestamp": "2014-05-14T21:41:58.171"
}
		\end{lstlisting}




\section{Backend}
	Clusterify è un'applicazione Django\cite{django_project} costruita con l'idea di essere pluggabile, ovvero con la possibilità di inserire, con poco lavoro, nuovi \emph{reader} e nuovi algoritmi di clustering. Si può vedere il tutto come una struttura a \emph{Lego} dove si possono cambiare dei blocchi in modo da ottenere il risultato migliore, senza necessariamente modificare il codice del \emph{core}. Una volta stabiliti questi blocchi, clusterify farà il resto per calcolare i clusters dai dati dati in input.

	\subsection{Reader}
		Un reader è un oggetto che permette di caricare all'interno del workflow un insieme di testi da analizzare.

		Per mezzo di questo oggetto è possibile leggere dati da qualsiasi fonte, che sia un social network, come nel nostro caso, da un file in memoria, piuttosto che da un database, implementando solamente il metodo $\_texts()$.

		Questo metodo deve ritornare una lista di ennuple; ognuna di queste deve contenere testo, url (se esiste) e l'autore (se esiste).

		\subsubsection{TwitterReader}
			TwitterReader, come dice il nome, è il reader che è stato implementato per poter ottenere i dati da Twitter; $\_texts()$, in questo caso, fa una richiesta a Twitter e, una volta ricevuti, formalizza i dati.

	\subsection{Algoritmi di clustering}
		Un algoritmo di clustering, come visto in precedenza, è un algoritmo che permette di raggruppare oggetti all'interno di un insieme, mantenendo uniti quelli che secondo alcune regole risultano più coesi.

		Come per il \emph{reader}, anche questi algoritmi possono essere estesi con nuovi codici che implementano regole diverse, in modo da poter soddisfare qualsiasi richiesta. Clusterify nasce con tre algoritmi di clustering nel core, utili per valutare quale performava meglio nella suddivisione di topic definiti utilizzando dataTXT.

		\subsubsection{K-Means}
			K-means è il nome di un algoritmo di clustering che punta a creare $k$  gruppi distinti di uguale varianza, minimizzando l'inerzia del gruppo. Questo algoritmo richiede che il numero di cluster sia definito a priori. 

L'algoritmo punta a scegliere $k$ centroidi $C$ che minimizzano lo scarto quadratico medio con un dataset $X$ di $n$ elementi $(x_1,..., x_n)$, creando $k$ insiemi $S = {S_1, ..., S_K}$:

\begin{equation*}
	\underset {\mathbf{S}} {\operatorname{arg\,min}}  \sum_{i=1}^{k} \sum_{\mathbf x_j \in S_i} || x_j - \mu_i ||^2 
\end{equation*}

L'algoritmo in se è composto da tre passi:
\begin{enumerate}
	\item Scelta dei centroidi iniziali;
	\item Assegnamento di ogni componente;
	\item Creazione di nuovi centroidi;
\end{enumerate}

Dopo aver scelto i primi $k$ centroidi dalle componenti del dataset $X$, l'algoritmo continua ad eseguire due operazioni finché i centroidi proposti possono essere considerati corretti, ovvero che la propria posizione non venga modificata con il passare delle iterazioni. 

L'assegnamento di ogni componente ad un centroide si esegue utilizzando la distanza euclidea al quadrato, questa sarà assegnata al punto medio più vicino; si può calcolare la distribuzione delle componenti in questo modo:
\begin{equation*}
	S_i^{(t)} = \big \{ x_p : || x_p - m^{(t)}_i ||^2 \le || x_p - m^{(t)}_j ||^2 \ \forall j, 1 \le j \le k \big\}
\end{equation*}
dove ogni $x_p$ è assegnato a esattamente un $S^{(t)}$, anche se idealmente potrebbe essere assegnato a più di un centroide.

La creazione dei nuovi centroidi utilizza, invece, il valore medio di ogni componente appartenente ad ogni centroide precedentemente definito.
\begin{equation*}
	m^{(t+1)}_i = \frac{1}{|S^{(t)}_i|} \sum_{x_j \in S^{(t)}_i} x_j 
\end{equation*}

Per quanto riguarda la scelta dei centroidi ci si può basare sul caso, scegliendo in modo \emph{random} i punti, con l'idea che con il passare delle iterazioni queste stime si sistemeranno; la seconda scelta vede il posizionamento dei centroidi in modo equidistante dagli tra loro, col fine di cercare di massimizzare l'area coperta e minimizzare il numero di iterazioni necessarie per giungere nella situazione finale di stallo.

		\subsubsection{Spectral}
			Spectral è il nome di un algoritmo di clustering che punta ad utilizzare gli autovalori della matrice delle adiacenze per capire come partizionare gli elementi dell'insieme.

Dato un grafo, rappresentato appunto dalla matrice delle adiacenze, l'algoritmo cerca il taglio minimo all'interno del grafo ed elimina gli archi che passano dal primo insieme al secondo. Questo procedimento continua in modo iterativo finche non si raggiungono, come nel caso del k-means $k$ cluster.s

Lo spectral basa il suo funzionamento sulla matrice $A$ della adiacenze, simmetrica per definizione, con $A_{ij}\geq 0$ rappresentante la similarità tra l'elemento in indice $i$ e quello in indice $j$; questa matrice, conosciuta con il nome di matrice normalizzata di Laplace, viene definita in questo modo:

\begin{equation*}
	L^{norm} = I-D^{-1/2} A D^{-1/2},
\end{equation*}
dove $D$ è la matrice diagonale costruita in questo modo:
\begin{equation*}
	D_{ii} = \sum_j A_{ij}.
\end{equation*}

L'algoritmo partiziona gli elementi in due insiemi distinti ($B_1$, $B_2$) basando la scelta del taglio sui valori dell'autovettore $v$ relativo al secondo più piccolo autovalore calcolato su $L^{norm}.

		\subsubsection{Affinity Propagation}
			Affinity Propagation è un algoritmo di clustering basato sul concetto di "passaggio di messaggi" tra le varie componenti. Una caratteristica fondamentale di quest'algoritmo è la mancanza di necessità di determinare a priori la quantità di cluster che saranno necessari.

Un dataset è descritto utilizzando degli esemplari identificati come le componenti più rappresentative dell'insieme. I messaggi trasmessi tra coppie sono utilizzati per calcolare l'idoneità che una componente ha di essere l'esemplare dell'altra. Quest'ultima sarà successivamente aggiornata con i valori provenienti dalle altre coppie. Questo processo viene ripetuto iterativamente finché il tutto non converge, dando vita al cluster.

Sia $(x_1, ..., x_n)$ l'insieme dei punti da clusterizzare e sia $S$ una funzione che permette di definire la similarità di due punti.

L'algoritmo procede alternando due fasi di "passaggio di messaggi", aggiornando due matrici:

\begin{itemize}
	\item La matrice di responsabilità $R$ ha valori $r(i, k)$ che quantificano quanto $x_k$ sia ben situato nel cluster contenente $x_i$, prendendo in considerazione tutti gli altri elementi contenuti in questo insieme;
	\item La matrice di disponibilità $A$ contiene i valori $a(i, k)$ che rappresentano quanto sia appropriato per $x_i$ scegliere $x_k$ come prossimo elemento del cluster a cui lui stesso appartiene.
\end{itemize} 

Entrambe queste matrici sono inizializzate a $0$ e, successivamente, saranno aggiornate in modo iterativo con le seguenti espressioni:
\begin{align*}
	r(i,k) &= s(i,k) - \max_{k' \neq k} \left\{ a(i,k') + s(i,k') \right\}\\
	a(i,k) &= \min \left(0, r(k,k) + \sum_{i' \not\in \{i,k\}} \max(0, r(i',k)) \right) for i \neq k\\
	a(k,k) &= \sum_{i' \neq k} \max(0, r(i',k))
\end{align*}

Il processo terminerà quando il tutto raggiungerà, come nel K-means, uno stato di fermo\cite{affinity}.

		
\section{Scelta dell'algoritmo di clustering}
	Una volta definiti i quattro algoritmi è necessario capire quale performa meglio nel clusterizzare argomenti secondo la relatedness offerta da dataTXT.

È stato chiesto a dieci persone di creare dei cluster partendo dalle entità estratte dagli ultimi 50 tweet che popolavano il loro newsfeed e il loro prodotto è stato paragonato con l'output di tutti gli algoritmi proposti. 

Per questo compito è stata utilizzata la funzione ``adjusted\_rand\_score'' offerta da scikit-learn -- \url{http://scikit-learn.org/stable/modules/generated/sklearn.metrics.adjusted_rand_score.html} -- che permette di calcolare la similarità tra i cluster proposti e quello risultante. Per fare questo si considerano tutte le coppie di elementi e si contano quelle che sono state assegnate allo stesso o ad un altro cluster.

Per permettere agli intervistati di formare questi cluster è stato creato un apposito file su google drive -- \url{https://docs.google.com/spreadsheets/d/1RNte6_I-cO5A23dKoaAyIRX5iWOGAEN3S25VIpXQLWU/edit?usp=sharing} -- con gli argomenti da clusterizzare pre-caricati ed è stato chiesto loro di segnare con lo stesso numero quelli che secondo loro avrebbero dovuto appartenere allo stesso cluster. 

Lanciando \texttt{python manage.py analyze} dalla \emph{root} di Clusterify questi dati vengono analizzati. I risultati sono riportati nel grafico:

\catcode`\_=12
\begin{tikzpicture}
	\begin{axis} [
   	 	ybar,
		width=12cm,
	        	height=8cm,
        		ymin=0,
	  	ymax=150,  
		bar width=3pt,
        		legend columns=2,
    		legend style={draw=none,nodes={inner sep=3pt}},,
    		ylabel={\% successo},
    		symbolic x coords={
    			MartinBrugrara,
    			AndreaDellera,
    			gi4n1uc4,
    			apheniti,
    			john_frigo,
		    	edorigatti,
			cosaunlama,
			michelaelle,
			SpiritualGuru,
			carlotta_93
    		},
    		xtick=data,
    		nodes near coords,
    		nodes near coords align={vertical},
		nodes near coords={
			\pgfmathprintnumber[fixed zerofill, precision=2]{\pgfplotspointmeta}
		},
		every node near coord/.append style={
			font=\tiny,
			anchor=west,
                		rotate=90
        		},
		x tick label style={rotate=45, anchor=east},
    	]
		\addplot coordinates {
			(MartinBrugrara, 89.61) 
			(AndreaDellera, 71.72) 
			(gi4n1uc4, 85.81)
			(apheniti, 77.02)
			(john_frigo, 94.47)
			(edorigatti, 81.56)
			(cosaunlama, 79.00)
			(michelaelle, 76.43)
			(SpiritualGuru, 81.04)
			(carlotta_93, 74.23)
		};
		\addplot coordinates {
			(MartinBrugrara, 57.08) 
			(AndreaDellera, 55.56) 
			(gi4n1uc4, 56.89)
			(apheniti, 64.60)
			(john_frigo, 59.87)
			(edorigatti, 57.96)
			(cosaunlama, 58.86)
			(michelaelle,  57.15)
			(SpiritualGuru, 56.46)
			(carlotta_93, 65.06)
		};
		\addplot coordinates {
			(MartinBrugrara, 64.26) 
			(AndreaDellera, 67.84) 
			(gi4n1uc4, 65.23)
			(apheniti, 64.77)
			(john_frigo, 61.94)
			(edorigatti, 61.73)
			(cosaunlama, 67.00)
			(michelaelle,  64.50)
			(SpiritualGuru, 62.50)
			(carlotta_93, 71.69)
		};
		\addplot coordinates {
			(MartinBrugrara, 66.82) 
			(AndreaDellera, 63.63) 
			(gi4n1uc4, 64.32)
			(apheniti, 64.44)
			(john_frigo, 63.59)
			(edorigatti, 62.81)
			(cosaunlama, 67.00)
			(michelaelle,  64.33)
			(SpiritualGuru, 62.78)
			(carlotta_93, 72.88)
		};
		\legend{Affinity Propagation, Specral, K-means, Hierarchical}
	\end{axis}
\end{tikzpicture}
\catcode`\_=8

Come si può dedurre dal grafico, ogni persona che ha creato il proprio cluster ideale si è avvicinata inconsciamente alla soluzione proposta dall'\emph{Affinity Propagation}. 

Una volta raccolti i dati è stato mostrato al campione intervistato il risultato proposto su di un nuovo dataset, generato sempre dall'account Twitter dei diretti interessati, ed il risultato nel 90\% dei casi, era sensato ed utilizzabile.

Alla luce di questi risultati, Clusterify utilizzerà Affinity Propagation.


\section{Workflow}
	Clusterify non è altro che un sistema di \emph{pipe}, dove un comando viene eseguito prendendo come input l'output del precedente.

	Come si è visto nel sezione relativa al \emph{processo di caricamento} nel capitolo \emph{frontend}, il workflow si può suddividere in tre parti:

	\begin{enumerate}
  		\item Acquisizione dei testi
  		\item Annotazione dei testi
 		\item Clustering dei testi
	\end{enumerate} 
	
	\subsection{Acquisizione dei testi}
		La parte di acquisizione testi, come detto in \emph{reader} è una delle due parti pluggabili di Clusterify.

		Questa fase si occupa del prendere i dati da una qualsivoglia fonte e di salvarli in modo conferme allo standard creato. All'interno di questa parte è possibile implementare, se necessario, un livello di cache in modo da impedire di richiedere più volte le stesse informazioni.

		In Clusterify, \emph{TwitterReader} utilizza \emph{Twython}, una libreria Python che offre un accesso facile ai dati presenti su Twitter\cite{twython}. In altre parole, Twython offre un \emph{wrapper} alle API di Twitter, permettendo al programmatore di ottenere dati senza preoccuparsi della gestione errori, dell'aggiornamento delle API e quant'altro. 

	\subsection{Annotazioni dei testi}
		La seconda fase del workflow si occupa di annotare i testi per mezzo di dataTXT.
		
		Quest'operazione è fondamentale per un buon risultato finale in quanto questa può far incorrere nel riconoscimento di entità sbagliate; oppure, nel caso contrario, ci si può imbattere nell'estrazione di poche entità, ma molto precise. Avere troppe annotazioni sbagliate possono condurre nell'avere dei dati non veritieri; il contrario potrebbe portare a non avere nemmeno un entità estratta per testo analizzato e quindi, in fase di clustering, questi verrebbero persi.

		Al programmatore è quindi chiesto di scegliere molto attentamente i parametri che verranno passati all'estrattore, per non finire in uno dei due casi.

	\subsection{Clustering dei testi}
		L'ultima fase la si può pensare suddivisa a sua volta in tre sotto fasi:

		\begin{enumerate}
  			\item Creazione matrice delle adiacenze
  			\item Esecuzione dell'algoritmo di clustering
 			\item Uniformazione output
		\end{enumerate} 

		\subsubsection{Creazione matrice di adiacenza}
			La maggior parte degli algoritmi di clustering necessita di un grafo pesato su cui operare. Questo è rappresentato per mezzo di una matrice delle adiacenze $N \times N$, ove $N$ è la cardinalità dell'insieme composto da tutte le entità, e viene costruita in questo modo:

			\begin{equation*}
				Adj(a, b) = \begin{cases} 
					rel(a,b), & \mbox{se } a \neq b \\ 
					0, & altrimenti 
				\end{cases}
			\end{equation*}
			ove $Adj$ è la nostra matrice delle adiacenze, $a$ e $b$ sono i due topic e $rel(a,b)$ è la chiamata all'API dataTXT-REL che permette di capire quanto due entità sono correlate tra di loro.
			
			Si deduce che $Adj$ è una matrice speculare, con valori reali compresi tra $0$ e $1$, e che presenta $0$ sulla diagonale.

		\subsubsection{Esecuzione dell'algoritmo di clustering}
			Una volta generata la matrice delle adiacenze si può eseguire un algoritmo di clustering. Come già detto Clusterify permette di estendere gli algoritmi già presenti, quali \emph{Star}, \emph{Spectral}, \emph{Affinity Propagation}. Una volta stabilito l'algoritmo da utilizzare, Cluterify passerà a questo i vari dati e aspetterà l'esecuzione e il successivo output.

		\subsubsection{Uniformazione output}
			Dato l'output dell'algoritmo di clustering, i dati saranno processati nuovamente attraverso una funzione votata a due principali funzioni:
			\begin{itemize}
  				\item Rendere l'output conforme con il modello prestabilito
	  			\item Aggiungere informazioni ad ogni componente del cluster
 			\end{itemize} 
			
			La prima si occuperà di portare la struttura dati uscente dall'algoritmo in quella stabilita da Clusterify; la seconda, invece, permette di aumentare l'informazione che questa contiene, aiutandosi con dati calcolati in precedenza o sul posto.

\section{Frontend}
	Frontend
