Il metodo Ward si basa sul clustering gerarchico di tipo agglomerativo. Ad ogni passo dell'esecuzione si unisce la coppia di cluster che possiede la distanza minima. 

Questo metodo viene implementato dall'algoritmo Lance–Williams. Questo utilizza un approccio ricorsivo per aggiornare ad ogni passo la distanza tra i cluster, unendo i due gruppi più vicini.

Poiché non esiste un unico algoritmo basato su Ward, si può definire una famiglia di algoritmi, basati sullo stesso metodo, per calcolare la distanza tra due cluster. Sia
\begin{equation*}
	d_{(ij)k} = \alpha_i d_{ik} + \alpha_j d_{jk} + \beta d_{ij} +  \gamma |d_{ik} - d_{jk}|,  
\end{equation*}
la funzione generica che permette di calcolare questa distanza tra due cluster. Sia $d_{ij}$ la distanza tra l'$i$-esimo e il $j$-esimo cluster; $d_{(ij)k}$ la distanza tra $C_i \cup C_j$ e $C_k$; e siano $\alpha_i$, $\alpha_j$, $\beta$ e $\gamma$ i parametri caratterizzanti gli algoritmi che compongono questa famiglia.

Ad esempio l'algoritmo che viene utilizzato da Clusterify definisce
\begin{equation*}
	\alpha_i = \alpha_j  = \frac{n_l+n_k}{n_i+n_j+n_k}
\end{equation*}

\begin{equation*}
	\beta =\frac{-n_k}{n_i+n_j+n_k}
\end{equation*}

\begin{equation*}
	\gamma = 0
\end{equation*}
ed è conosciuto con il nome di \emph{Ward's minimum variance method}\cite{ward}.